\documentclass[french]{article}

\usepackage{microtype}
\usepackage{fontspec}
\setmainfont{Fira Sans}
\usepackage[a4paper]{geometry}
\usepackage{babel}

\title{Présentation du projet \textsc{Inalaya}}
\date{}
\begin{document}
    \maketitle
    \section*{Présentation}
    \textsc{Inalaya} vise à proposer une expérience sensible par l’ajout d’une couche sonore interactive sur 
    un objet de référence : image, objet matériel, contenu visuel numérique, ou environnement.
    Cette couche audio interactive est gérée par un scénario complexe : prise en compte 
    des  actions des utilisateurs, contraintes temporelles, contraintes spatiales, conditions, 
    embranchements, boucles, multi-entrées. 
    Ce scénario peut être créé par une ou plusieurs personnes en prenant en compte les caractéristiques de l’objet de référence.
    
    Concernant l’écriture et l’exécution de scénarios complexes, des travaux de recherche ont eu lieu 
    dans le cadre du projet \textsc{OSSIA} (Open Scenario System for Interactive Applications) soutenu 
    par l’agence nationale de la recherche.
    L’un des objectifs de ce projet a été d’offrir aux développeurs des outils génériques pour l’écriture de scénarios d’interactions complexes et l’exécution de ceux-ci, notamment via le développement du logiciel \textsc{i-score}\footnote{www.i-score.org}.
    
    En amont du projet OSSIA, Blue Yeti a effectué un premier travail sur la création de scénarios 
    audio basés sur des paramètres d’espace, utilisant la géolocalisation de l’utilisateur pour 
    naviguer dans un scénario s’exécutant sur terminaux mobiles iOS. 
    
    \subsection*{Blue Yeti}
    Blue Yeti conçoit et développe des dispositifs interactifs multimédia, visuels et sonores, dédiés à des usages culturels, éducatifs et artistiques.
    Blue Yeti intervient auprès des scénographes, enseignants, musées, centres scientifiques, collectivités, laboratoires de recherche, artistes et centres de création pour la conception et le développement d'installations multimédia et d’espaces immersifs interactifs, basés sur des technologies telles que interfaces tactiles multitouch, interaction sans contact, réalité augmentée, périphériques mobiles.
    Créée en 2007, l'entreprise est actuellement constituée d’une équipe de 6 personnes permanentes et d’un réseau de développeurs indépendants, designers sonores et artistes visuels qui apportent leur créativité et leur savoir-faire technique en fonction des spécificités de chaque projet.
    
    Blue Yeti possède également des compétences dans le domaine de la captation et l’analyse temps 
    réel du geste, ainsi que dans le mapping entre geste et processus sonores. Ces compétences ont 
    été acquises durant la thèse de Jean-Michel Couturier, co-gérant de Blue Yeti (thèse effectuée au 
    laboratoire de Mécanique et d’Acoustique de Marseille, soutenue en 2004) et consolidées par la suite dans différents projets de l’entreprise.
    
    \subsection*{Enjeux}
    
    \section*{Démonstrateur}
    \begin{itemize}
    \item bosch
    \item décliner éléments qui sont interactifs
    \item quelle interaction 
    \end{itemize}
    
    \section*{Méthode}
    Outils:  
    \begin{itemize}
    \item i-score pour trame scénaristique générale
    \item Ableton pour gestion du son
    \item Unity pour gestion du monde et des relations physiques entre objets
    \end{itemize}
    
    \subsection*{Écriture et création}
    \begin{itemize}
    \item écriture de l'interaction
    \item scénarisation
    \item écriture spatiale -> dessins de zone
    \item écriture audio ->
    \item screenshot du scénario
    \end{itemize}
    
    \subsection*{Jeu et interaction}
    \begin{itemize}
    \item description des interactions possibles
    \item descriptions des mappings
    \item screenshot de l'install
    \end{itemize}
    
    \subsection*{Prototypage}
    \begin{itemize}
    \item dans unity ? 
    
    \item dans i-score ? 
    \end{itemize}
    
    \subsection*{Ambitions}
    \begin{itemize}
    \item World building
    \item Écriture de trajectoires spatiales complètes
    \item Écriture 
    \item Intégration du séquenceur audio
    \item Objets sonores interactifs haut et bas niveau
    \end{itemize}
\end{document}
